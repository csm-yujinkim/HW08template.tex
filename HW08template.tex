%=============================================================================
\documentclass[12pt]{article}
\usepackage{latexsym}
\usepackage{graphicx}
\usepackage{booktabs}
\usepackage{multirow}
\usepackage{amsmath}
\usepackage{amsfonts}
\usepackage[caption=false]{subfig}
\usepackage{enumerate}

%=============================================================================
\setlength{\evensidemargin}{-0.25in}
\setlength{\oddsidemargin} {-0.25in}
\setlength{\textwidth}     {+7.00in}
\setlength{\topmargin}     {+0.00in}
\setlength{\textheight}    {+8.50in}
%=============================================================================
\makeatletter
%\renewcommand{\baselinestretch}{1.2}

\normalsize
%=============================================================================

%==============================================================================
\pagestyle{plain}
%
\date{}
\begin{document}
	%==============================================================================
	\begin{flushleft}
		\large \bf
		Homework 8 \\
	\end{flushleft}
	%==============================================================================
	{\bf
		Please note that you have to typeset your assignment using \LaTeX.
		%
		Hand-written assignment will not be graded.
		%
		You need to submit a PDF version on Canvas by 23:59, April 28.
		%
	}
	
	
	\begin{enumerate}

 
		
		\item ($5 \times 2$)
		For each of the following binary relations $R$ on $\mathbb{N}$, decide which of the given ordered pairs belong to $R$.
		\begin{enumerate}[a.]
			\item
			$R = \{(x,y) \mid x + y < 7 \}$;\\
			$\left(1, 3\right), \left(2, 5\right), \left(3, 3\right), \left(4, 4\right)$
			
			\item
			$R = \{(x,y) \mid y \text{ is a perfect square}\}$; \\
			$\left(1, 1\right), \left(4, 2\right), \left(3, 9\right), \left(25, 5\right)$			
		\end{enumerate}
		{\bf Solution:}
		
		\begin{enumerate}[a.]
			\item
			\begin{align*}
				1 + 3 = 4 &< 7 \\
				3 + 3 = 6 &< 7
			\end{align*}

			$\left(1, 3\right)$ and $\left(3, 3\right)$ belong to $R$.

			\begin{align*}
				2 + 5 = 7 &\not< 7 \\
				4 + 4 = 8 &\not< 7
			\end{align*}

			$\left(2, 5\right)$ and $\left(3, 3\right)$ do not belong to $R$.

			\item
			\begin{align*}
				1 &= 1^2 \\
				9 &= 3^2
			\end{align*}

			$\left(1, 1\right)$ and $\left(3, 9\right)$ belong to $R$.

			\begin{align*}
				\sqrt{2} \approx 1.414 \\
				\sqrt{5} \approx 2.236
			\end{align*}

			$\left(4, 2\right)$ and $\left(25, 5\right)$ do not belong to $R$.
		\end{enumerate}
		\newpage
		\item ($10  \times 2$)		
		Test the following binary relations on the given sets $S$ for reflexivity, symmetry, antisymmetry, transitivity.
		\begin{enumerate}
			\item
			$S = \mathbb{Q}$
			
			$R = \{(x,y) \mid |x| \leq |y|\}$
			
			\item
			$S = \mathbb{N}$
			
			$R = \{(x, y) \mid x \times y\text{ is even} \}$
			
		\end{enumerate} 
		{\bf Solution:}
		
		\begin{enumerate}[a.]
			\item
				\begin{itemize}
					\item Reflexivity: Let $a$ be a rational number ($\mathbb{Q}$).
					Then, $a=a$. $|a| = |a|$.
					$|a| \le |a|$.
					So, $(a, a)$ belongs to $R$. $R$ has reflexivity.
					\item Symmetry: $(1, 100)$ belongs to $R$ because $|1| \le |100|$.
					However, $(100, 1)$ does not because $|100| \not\le |1|$.
					Therefore, $R$ does not have symmetry.
					\item $(1, -1)$ belongs to $R$ because $|1| = |-1|$, and so does $(-1, 1)$ for the same reason.
					However, $1 \ne -1$, so $R$ does not have antisymmetry.
					\item Transitivity: Let $(a, b)$ and $(b, c)$ belong to $R$.
					Then, $|a| \le |b|$ and $|b| \le |c|$.
					By a property of $\le$, $|a| \le |c|$. So, $R$ has transitivity.
				\end{itemize}

			\item
				\begin{itemize}
					\item Reflexivity: $1$ is a nonnegative integer ($\mathbb{N}$), but $1\times1=1$, and is not even.
					$R$ does not possess reflexivity.
					\item Symmetry: If $x\times y$ is even, then $y\times x$ must also be, because $x\times y=y\times x$.
					So, $R$ possesses symmetry.
					\item Antisymmetry: $2\times 4$ is even, and $4\times 2$ is also even.
					However, $2\ne 4$.
					$R$ does not possess antisymmetry.
					\item Transitivity: $1\times 2$ is even, and $2\times 3$ is also even.
					However, $1\times 3$ is not even.
					$R$ does not possess transitivity.
				\end{itemize}
			
			
		\end{enumerate}
		
		\newpage
		\item ($5 \times 4$)
		Identify each relation on $\mathbb{N}$ as one-to-one, one-to-many, many-to-one, or many-to-many.
		\begin{enumerate}[a.]
			\item
			$R = \{\left(1, 2\right), \left(1, 4\right), \left(1, 6\right), \left(2, 3\right), \left(4, 3\right)\}$
			\label{item:3a}
			
			\item
			$R = \{\left(9, 7\right), \left(6, 5\right), \left(3, 6\right), \left(8, 5\right)\}$
			\label{item:3b}
			
			\item
			$R = \{\left(12, 5\right), \left(8, 4\right), \left(6, 3\right), \left(7, 12\right)\}$
			\label{item:3c}
			
			\item
			$R = \{\left(2, 7\right), \left(8, 4\right), \left(2, 5\right), \left(7, 6\right), \left(10, 1\right)\}$
			\label{item:3d}
		\end{enumerate}
		
		
		{\bf Solution:}
		
		\begin{enumerate}[a.]
			\item In the set of preimages, $1$ appears many times.
			In the range, $3$ appears many times.
			Therefore, relation \ref{item:3a} is many-to-many.
			\item In the set of preimages, each preimage appears only once.
			In the range, $5$ appears twice.
			So, relation \ref{item:3b} is many-to-one.
			\item In the set of preimages, each preimage appears only once.
			In the range, each image appears only once.
			So, relation \ref{item:3c} is one-to-one.
			\item In the set of preimages, $2$ appears twice.
			In the range, each image appears only once.
			So, relation \ref{item:3d} is one-to-many.
		\end{enumerate}
	\newpage
	
	\item ($15+15+20$)
	Which functions are one-to-one? Which functions are onto? Which functions are bijective? Describe the inverse function for any bijective function.
	\begin{enumerate}[a.]
		\item
		$f: \mathbb{Z} \to \mathbb{N}$ where $f$ is defined by $f\left(x\right) = x^2 +1$
		
		\item
		$f: \mathbb{N} \to \mathbb{N}$ where $f$ is defined by $f\left(x\right) = \left\lbrace 
		\begin{array}{ll}
		x/2   & \mbox{if } x \mbox{ is even } \\
		x + 1 & \mbox{if } x \mbox{ is odd }  
		\end{array}
		\right.$
		
		\item
		$f: \mathbb{N} \to \mathbb{N}$ where $f$ is defined by $f\left(x\right) = \left\lbrace 
		\begin{array}{ll}
		x + 1 & \mbox{if } x \mbox{ is even } \\
		x - 1 & \mbox{if } x \mbox{ is odd }  
		\end{array}
		\right.$
	\end{enumerate}		
	
	{\bf Solution:} 
	% Write your answer here
	\begin{enumerate} [a.]
		\item
		\begin{itemize}
			\item One-to-one? If this function is one-to-one, then $f(s_1) = f(s_2)$ must imply $s_1 = s_2$.
			
			Let $s_1 = 1$ and $s_2 = -1$.
			Then, $f(s_1) = 1$ and $f(s_2) = 1$.
			Therefore, $f$ is not one-to-one.
			
			\item Onto? If this function is onto, then the codomain $\mathbb{N}$ must be a subset of the range.
			The range does not contain $0$, but the codomain does contain it.
			Therefore, $f$ is not onto.

			\item Bijective? This function is neither one-to-one nor onto, so it is not bijective.
		\end{itemize}
		
		\item
		\begin{itemize}
			\item One-to-one? Both $x=1$ and $x=4$ yields $f(x)=2$, so this is not one-to-one.
			\item Onto? The range of this function is the union of $E$ and $O$, where $E$ is the set of images where the preimage is even, and $O$ is that where the preimage is odd.
			$E=\mathbb{N}$, so the range and codomain are equal.
			The following is the proof.
			
			\begin{itemize}
				\item
				$E$ is defined as follows: $\{n/2 \mid n, n/2 \in \mathbb{N}\}$.
			
				\item
				It has been provided that every $n$ in $E$ is also in $\mathbb{N}$.
				So, $E\subseteq\mathbb{N}$.
	
				\item
				Now, let's prove that $\mathbb{N}\subseteq E$.
				This means that for all $x\in\mathbb{N}$, $x\in E$.
				If $x\in\mathbb{N}$, then $2x\in\mathbb{N}$ and is an even number.
				Since $x=(2x)/2$, $x$ is in $E$.
				Every element of $\mathbb{N}$ is in $E$, so $\mathbb{N}\subseteq E$.
	
				\item
				In conclusion, $E = \mathbb{N}$.
	
				\item
				So, $E\cup O$, which is the range, is exactly the same as the codomain, $\mathbb{N}$.
			\end{itemize}
			
			\item Bijective? Since $f$ is not one-to-one, it is not bijective.
			
		\end{itemize}
		
		\item
		\begin{itemize}
			\item One-to-one? Yes. Proof that $f(s_1) = f(s_2)$ implies that $s_1 = s_2$: (Assume that $n=f(s_1)=f(s_2)$)
			\begin{itemize}
				\item Case 1: $n$ is odd. Solving $n = s_1 - 1$ for $s_1$, we get $s_1 = n + 1$. Similarly, $s_2 = n + 1$. Therefore, $s_1 = s_2$.
				\item Case 2: $n$ is even. Solving $n = s_1 + 1$ for $s_1$, we get $s_1 = n - 1$. Similarly, $s_2 = n - 1$. Therefore, $s_1 = s_2$.
				\item $n$ is either odd or even since $n\in\mathbb{N}$.
				\item It is therefore proven that $f(s_1) = f(s_2)$ implies that $s_1 = s_2$.
			\end{itemize}
			\item Onto? Yes. Proof that the codomain ($\mathbb{N}$) is a subset of the range ($C$):
			\begin{itemize}
				\item Let $n\in\mathbb{N}$. $n$ is even, or $n$ is odd.
				\begin{itemize}
					\item If $n$ is even, then $n+1$ is odd and is in $\mathbb{N}$. $f(n+1) = n$, so $n$ is an image of $f$.
					\item If $n$ is odd, then $n-1$ is even. Since $n>0$, $n-1\ge0$, and thus $n-1\in\mathbb{N}$. $f(n-1) = n$, so $n$ is an image of $f$.
				\end{itemize}
				\item $n$ is an image of $f$, so if $n\in\mathbb{N}$ then $n\in C$.
				\item This is to say that $\mathbb{N}\subseteq C$.
			\end{itemize}
			\item Bijective? Since $f$ is both one-to-one and onto, it is bijective.
		\end{itemize}
	\end{enumerate}
		
	\end{enumerate}
\end{document}
%==============================================================================